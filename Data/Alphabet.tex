\documentclass{article}
\usepackage{graphicx}
\usepackage{amsmath} % For \text command

\begin{document}

% First Image
\begin{figure}[htbp]
\centering
\includegraphics[width=\textwidth]{dakota_dictionary_page1.png} % Replace with your image filename
\caption{A Dakota-English Dictionary, Page 1}
\label{fig:page1}
\end{figure}

\noindent
A DAKOTA-ENGLISH DICTIONARY.

\noindent
BY STEPHEN R. RIGGS.

\vspace{0.5cm}
\noindent
THE ALPHABET.

\vspace{0.5cm}
\noindent
\textsc{VOWELS.}

\noindent
The vowels are five in number, and have each one uniform sound, except when followed by the nasal ``n,'' which somewhat modifies them.

\begin{tabular}{ll}
    a & has the sound of English \textit{a} in \textit{father}.\\
    e & has the sound of English \textit{e} in \textit{they}, or of \textit{a} in \textit{face}.\\
    i & has the sound of \textit{i} in \textit{marine}, or of \textit{e} in \textit{me}.\\
    o & has the sound of English \textit{o} in \textit{no}, \textit{note}.\\
    u & has the sound of \textit{u} in \textit{rule}, or of \textit{oo} in \textit{food}.
\end{tabular}

\vspace{0.5cm}
\noindent
\textsc{CONSONANTS.}

\noindent
The consonants are twenty-four in number, exclusive of the sound represented by the apostrophe (‘).

\begin{tabular}{ll}
 b  & has its common English sound.\\
 $\check{c}$  & is an aspirate with the sound of English \textit{ch}, as in \textit{chin}.  In the \\
    & Dakota Bible and other printing done in the language, it\\
    & has not been found necessary to use the diacritical mark.*\\
 $\acute{c}$  & is an emphatic $\check{c}$. It is formed by pronouncing ``$\check{c}$'' with a \\
    & strong pressure of the organs, followed by a sudden expul- \\
    & sion of the breath.\textdagger\\
 d  & has the common English sound.\\
 g  & has the sound of \textit{g} hard, as in \textit{go}.\\
 ġ & represents a deep sonant guttural resembling the Arabic \textit{ghain} \\
    & ($\dot{\varepsilon}$).  Formerly represented by \textit{g} simply.\textdagger\\
 h  & has the sound of \textit{h} in English. \\
 ḣ & represents a strong surd guttural resembling the Arabic \textit{kha} ($\dot{\text{\reflectbox{$\tau$}}}$). \\
    & Formerly represented by $\eta$.\textdagger
\end{tabular}

\vspace{0.5cm}
\noindent
\footnotesize{* For this sound Lepsius recommends the Greek $\chi$. \hspace{0.2cm} \textdagger These are called \textit{cerebrals} by Lepsius.}\\
\footnotesize{\textdagger These correspond with Lepsius, except in the form of the diacritical mark.}

\vspace{0.5cm}
\noindent
VOL. VII.---1


% Second Image
\begin{figure}[htbp]
\centering
\includegraphics[width=\textwidth]{dakota_dictionary_page2.png} % Replace with your image filename
\caption{A Dakota-English Dictionary, Page 2}
\label{fig:page2}
\end{figure}


\noindent
2 \hspace{1cm} DAKOTA-ENGLISH DICTIONARY.

\begin{tabular}{ll}
    k & has the same sound as in English. \\
    $\underset{\raisebox{0.3ex}{\rlap{\textvisiblespace}{\textvisiblespace}}}{\text{k}}$ & is an emphatic letter, bearing the same relation to \textit{k} that ``$\acute{c}$'' \\
       & does to ``$\check{c}$.''  In all the printing done in the language, it is \\
       & still found most convenient to use the English \textit{q} to repre- \\
       & sent this sound.* \\
    l & has the common sound of this letter in English. It is peculiar \\
        & to the \textit{Titonwan} dialect. \\
    m & has the same sound as in English. \\
    n & has the common sound of \textit{n} in English. \\
    ŋ & denotes a nasal sound similar to the French \textit{n} in \textit{bon}, or the \\
       & English \textit{n} in \textit{think}. As there are only comparatively very \\
       & few cases where a full \textit{n} is used at the end of a syllable, no \\
       & distinctive mark has been found necessary. Hence in all \\
       & our other printing the nasal continues to be represented by \\
       & the common \textit{n}. \\
    p & has the sound of the English \textit{p}, with a \textit{little more volume} and \\
      & \textit{stress of voice}. \\
    $\underset{\raisebox{0.3ex}{\rlap{\textvisiblespace}{\textvisiblespace}}}{\text{p}}$ & is an emphatic, bearing the same relation to \textit{p} that ``$\acute{c}$'' does \\
       & to ``$\check{c}$''.*\\
    s & has the surd sound of English \textit{s}, as in \textit{say}.\\
    ṡ & is an aspirated \textit{s}, having the sound of English \textit{sh}, as in \textit{shine}. \\
       & Formerly represented by \textit{z}.\textdagger \\
    t & is the same as in English, with a little more volume of voice.\\
    $\underset{\raisebox{0.3ex}{\rlap{\textvisiblespace}{\textvisiblespace}}}{\text{t}}$ & is an emphatic, bearing the same relation to \textit{t} that ``$\acute{c}$'' does \\
       & to ``$\check{c}$''.* \\
    w & has the power of the English \textit{w}, as in \textit{walk}.\\
    y & -has the sound of English \textit{y}, as in \textit{yet}.\\
    z & has the sound of the common English \textit{z}, as in \textit{zebra}. \\
    ż & is an aspirated \textit{z}, having the sound of the French \textit{j}, or the English \\
      & \textit{s} in \textit{pleasure}. Formerly represented by \textit{j}.\textdagger \\
\end{tabular}

\noindent
The apostrophe (') is used to denote an hiatus, as in \textit{ma'a}. It seems to be\\
\hspace*{1cm} analogous to the Arabic \textit{hamzeh} ($\text{\reflectbox{$\varepsilon$}}$).

\vspace{0.3cm}
\noindent
NOTE.---Some Dakotas, in some instances, introduce a slight \textit{b} sound before the \\
\textit{m}, and also a \textit{d} sound before \textit{n}. For example, the preposition ``om,'' \textit{with}, is by some \\
persons pronounced ``obm,'' and the preposition ``en,'' \textit{in}, is sometimes spoken as if it \\
should be written ``edn.''  In these cases, the members of the Episcopal mission among \\
the Dakotas write the \textit{b} and the \textit{d}, as ``ob,'' ``ed.''

\vspace{0.5cm}
\noindent\footnotesize
{*These are called \textit{cerebrals} by Lepsius.}\\
\footnotesize{\textdagger These correspond with Lepsius, except in the form of the diacritical mark.}

\end{document}