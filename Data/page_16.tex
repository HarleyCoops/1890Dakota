\documentclass{article}
\usepackage{amsmath}

\begin{document}

\noindent 16 \hfill DAKOTA--ENGLISH DICTIONARY.

\vspace{0.5cm}

\noindent a-ho'-pe-ya, \textit{part.} honoring, respecting, observing. \textit{Adv.}, obediently.

\noindent a-ho'-taŋ, \textit{v. a.} to make a noise around one—ahowataŋ, ahomataŋ, ahountanpi.

\noindent a-ho-taŋ'-ka, \textit{n.} one who makes a noise around.

\noindent a-ho'-toŋ, \textit{v. a.} (a and hoton) to cry out for, as a bird for food.

\noindent a-ho'-toŋ-toŋ, \textit{v. red.} of ahoton; to cry out for, bawl for anything.

\noindent a-hu'-tkaŋ-yaŋ, \textit{adv.} branching, having many prongs or roots. See hutkan.

\noindent a-ham'-ya, \textit{v.} (a and hamya) to scare on, as game—ahamwaya.

\noindent a-haŋ'-haŋ, \textit{v. a.} to do a thing carelessly, not to have one's mind on it—awahanhan.

\noindent a-haŋ'-haŋ-ka, \textit{adj.} careless, negligent.

\noindent a-ha'-pa, \textit{v.} See ahbapa.

\noindent a-hba'-ya, \textit{adv.} mildly.

\noindent a-hba'-ye-daŋ, \textit{adv.} mildly, patiently: ahbayedan waun.

\noindent a-hćo', \textit{n.} the part of the arm above the elbow; that part of the wing of a fowl next the body.

\noindent a-hdah'-ye-će-śn, \textit{v.} to haunt about a place: i. q. amahyeća.

\noindent a-hdo', \textit{v. n.} (a and hdo) to growl over or about a thing, as a dog over a bone.

\noindent a'-he, \textit{v. n.} to evaporate: ahe aya. to decrease or fall, as the water in a river, lake, etc.

\noindent a-he'-waŋ-ka, \textit{v. n.} (a and hewanka) to be frost on anything.

\noindent a'-he-ya, \textit{v. a.} to cause to evaporate—ahewaya.

\noindent a'-ho, \textit{v. n.} to stand up or back, as hair on the forehead: ite aho.

\noindent a-hpa'-ya, \textit{v. n.} to fall upon—amahpaya. See ahiphpaya.

\noindent a-hpe'-ya, \textit{v. a.} to throw upon; to throw away; to leave, forsake—ahpewaya, ahpeunyanpi. See ehpeya, which is more commonly used.

\noindent a-hta'-ni, \textit{v.} (a and htani) to labor for one; to work on anything; to sin, break a law—ahtani, ayahtani, unkahtanipi, ayaćihtani.

\noindent a-hta'-ta, \textit{adj.} languid, feeble.

\noindent a-hta'-te-ća, \textit{adj.} weak, feeble.

\noindent a-htu'-daŋ, \textit{n.} something to be spit upon; i. q. ṡićedapi.

\noindent a-htu'-ta, \textit{adj.} a little thawed.

\noindent a-htu'-te-ća, \textit{adj.} a little thawed; thawing some.

\noindent a-i', \textit{v. a.} to carry or take to a place—awai, unkaipi; to charge with or lay upon, accuse, as en ai, en amai; to visit upon, as for a sin.

\noindent a-i', \textit{v.} col. pl. of i; they reached a place.

\noindent a-i'-a, \textit{v. a.} (a and ia) to talk about, consult in regard to; to speak evil of, slander—awaia, ayuia, unkaiapi, amaia, aniia, aćiia.

\noindent a-i'-a-pi, \textit{n.} consultation; slander.

\noindent a-i'-ća-ġa, \textit{v.} (a and ićaga) to grow on, yield, produce.

\noindent a-i'-ćah, \textit{v.} cont. of aićaga.

\end{document}